% Chapter 3

\section{引用与链接}

\subsection{脚注}
注释是对论文中特定名词或新名词的注解。注释可用页末注或篇末注的一种。选择页末注的应在注释与正文之间加细线分隔,线宽度为1磅,线的长度不应超过纸张的三分之一宽度。同一页类列出多个注释的,应根据注释的先后顺序编排序号。字体为宋体5号,注释序号以“\circled{1}、\circled{2}”等数字形式标示在被注释词条的右上角。页末或篇末注释条目的序号应按照“\circled{1}、\circled{2}”等数字形式与被注释词条保持一致。示例:这里有个注释\footnote{我是解释注释的}。

\subsection{引用文中小节}\label{sec:ref}
如引用小节~\ref{sec:ref}

\subsection{引用参考文献}
这是一个参考文献引用的范例\cite{Brogan1967Adjoint}

还可以采用上标的引用方式\upcite{Hu2016Analytical}

引用多个文献\cite{Bucco2015Adjoint,Chen2011Trajectory,CHEN2014Launch,Harpold1979Shuttle}

文献引用需要配合BibTeX使用,很多工具可以直接生成BibTeX文件(EndNote, NoteExpress, 百度学术,谷歌学术),此处不作介绍。

\subsection{链接相关}
模板使用了hyperref处理相关链接,使用\verb|href|可以生成超链接,链接周围的方框在打印时不会出现。可以在cls文件中修改相应的hypersetup项来关闭方框:\verb|\hypersetup{hidelinks}|。如果需要输出网址,可以使用\verb|url|包,示例:\url{https://github.com}。